\documentclass[8pt,a4paper]{beamer}
%\usepackage[utf8]{inputenc}
\usepackage[output-decimal-marker={,}]{siunitx}
\usepackage[compat=1.1.0]{tikz-feynman}
\usepackage{wrapfig}
\author{Adriano Del Vincio, (562946)}
\title{Measure of CP violation in B+/- meson}
\usetheme{Madrid}

\begin{document}

\frame{\titlepage}

\begin{frame}{Fisica del processo}
\framesubtitle{Studio dell'asimmetria CP nel decadimento in $3$ kaoni dei Mesono $B\pm$}

Lo studio dell'asimmetria $CP$ rappresenta una delle maggiori aree di indagine nella fisica delle alte energie. Diverse collaborazioni in passato (\textit{BaBar} presso lo SLAC, \textit{Belle} presso KEK) hanno misurato l'asimmetria nel comportamento tra materia/antimateria in differenti canali di decadimento, come i mesoni $B_{0}$ e $\overline{B_{0}}$. In questo progetto si è analizzato il decadimento dei mesoni carichi in  $3 K$, utilizzando dati collezionati a LHCb nel 2011.

\begin{figure}[hbtp]
\centering
\includegraphics[scale=0.5]{../Decay.pdf}
\caption{Decadimento del mesone $B+$ in tre kaoni, l'asimmetria è dovuta al cambiamento di flavour del quark $b$}
\end{figure}
\end{frame}

\begin{frame}{Dataset}
I dati a disposizione sono suddivisi in due TTree, che differiscono per l'orientazione del campo magnetico nell'esperimento, e sono analizzati separatamente. I due File contengono 25 variabili che descrivono la cinematica del processo.
I dati sono analizzati principalmente con \textit{RDataFrame}. L'analisi consiste nel selezionare gli eventi che provengono dal decadimento di interesse, rigettango eventi di fondo o le particelle che non possono essere identificate come Kaoni. Una volta selezionati gli eventi, si genera il Dalitz plot del decadimento e si rimuovono le risonanze che non sono di interesse. L'obiettivo dell'analisi è quello di ottenere una misura dell'asimmetria tra materia/antimateria nel decadimento, formalmente definita come:

\begin{equation}
A_{CP}(B^{\pm} \rightarrow f^{\pm}) = \dfrac{\Gamma(B^{+} \rightarrow f^{+}) - \Gamma(B^{-} \rightarrow f^{-})}{ \Gamma(B^{+} \rightarrow f^{+}) + \Gamma(B^{-} \rightarrow f^{-})}
\end{equation}
\end{frame}

\begin{frame}{Compute the invariant mass}
Per ricostruire il decadimento, è necessario calcolare la massa invariante dei mesoni $B$, a partire dalle impulso dei $3K$. La massa invariante è calcolata nello script \texttt{invmass.cpp}, utilizzando la funzione inline \textit{invMass}. Nella funzione si è esplicitato il modulo quadro del quadrimpulso totale:

\begin{align*}
P_{tot}^{\mu}P_{tot,\mu} = (k_{1} + k_{2} + k_{3})^{2}
\end{align*}

\begin{figure}[hbtp]
\centering
\includegraphics[width = \textwidth]{../Schermata del 2023-04-02 18-06-14.png}
\end{figure}

Per il calcolo, si è sfruttato la classe \texttt{ROOT::Math::LorentzVector}, che ha già implementato al suo interno i metodi per calcolare la massa invariante di un quadrivettore.
\end{frame}

\begin{frame}{DataSelection}

Sono definite anche altre quantità importanti, come energia, impulso del candidato $B$ a partire dall'energia e impulso dei kaoni. Si filtrano quegli eventi in cui una o più delle particelle è identificata come un muone. Infine si calcola la carica del candidato $B$

\begin{figure}[hbtp]
\centering
\includegraphics[width = \textwidth]{../rdf.png}
\end{figure}
\end{frame}

\begin{frame}{DataSelection}

Nello script \texttt{globalAsymmetry.cpp} si effettuano ulteriori tagli riguardanti l'identificazione delle particelle. Si escludono quelle particelle che hanno probabilità $> 50 \%$ di essere Pioni (che sono interessanti per un altro canale di decadimento) e si selezionano solo le particelle che hanno probabilità $> 50 \%$ di essere kaoni.

\begin{figure}[hbtp]
\centering
\includegraphics[width = 0.9\textwidth]{../Probabilities.pdf}
\end{figure}
\end{frame}

\begin{frame}{Fit mass B candidates}

Nello stesso script \texttt{globalAsymmetry.cpp} si effettua il fit alla massa invariante dei candidati $B$. Per Modellizzare il segnale si utilizza si utilizza la Cruijff function, una gaussiana asimmetrica nelle code, per tenere in conto dei fotoni emessi ISR e FSR. Il fondo è modellizzato dalla somma di un'esponenziale e da un'ARGUS function, che descrive il caso del decadimento in 4 corpi, di cui solo 3 sono effettivamente "visti" dai detector.

\begin{equation}
Argus(m,m_{0},c,p) = N \cdot m \cdot \Bigl[1 - \big(\frac{m}{m_{0}} \big)^{2} \Bigl]^{p} \cdot exp \Bigl[c \cdot \big(1 - (\frac{m}{m_{0}}\big)^{2}  \Bigl]
\end{equation}
\begin{equation}
Signal(N,\mu, \sigma_{L,R}, \alpha_{R,L}) = N \cdot exp \Bigl[  \dfrac{ - (x - \mu)^2}{2 \sigma_{L,R}^{2} + \alpha_{L,R} (x - \mu)^2  }	\Bigl]
\end{equation}

\end{frame}

\begin{frame}

\begin{figure}[hbtp]
\centering
\includegraphics[width = \textwidth]{../Invariant_Mass_Fit.pdf}
\end{figure}
\end{frame}

\begin{frame}
\begin{columns}
\column{0.6\textwidth}
\centering
Dal fit si ricavano i parametri che descrivono il segnale ed il fondo. Per eliminare gli eventi di fondo si è quindi selezionato un intervallo intorno al picco della massa invariante. Gli estremi dell'intervallo sono stati calcolati in \texttt{optimalCut.py}.
Massimizzando il rapporto tra segnale e rumore:

\begin{align*}
\dfrac{S(x,y)}{\sqrt{S(x,y)+B(x,y)}}
\end{align*}

Il codice calcola il numero aspettato di eventi di segnale e di fondo per una data coppia di punti (x,y). Alla fine si selezionano i valori che massimizzano la funzione sopra. 
\column{0.4\textwidth}
\begin{figure}
\includegraphics[width=0.6\textwidth]{../OptimalCut.pdf}
\caption{Frequency versus Voltage}
\end{figure}
\end{columns}









\end{frame}

\end{document}